\documentclass[a4paper,12pt]{article}
\usepackage[margin=2.54cm,left=3.5cm, marginparwidth=3.5cm, marginparsep=3mm, heightrounded]{geometry}
\usepackage{parskip}
\usepackage{color,graphicx}
\usepackage{etex}
\usepackage{textcomp}
\usepackage[main=british, russian]{babel}
\usepackage[autostyle]{csquotes}
\usepackage[usenames,dvipsnames]{xcolor}
\usepackage{titlesec}
\usepackage{calc}
\usepackage{multicol}
\usepackage{etoolbox}
\usepackage{hyperref}
\usepackage{hyphenat}
\urlstyle{tt}
\hypersetup{colorlinks,breaklinks=true,urlcolor=blue,linkcolor=blue,pdftitle={Pavel Iosad CV}}


\newlength\blockwidth
\setlength\blockwidth{3em}

\setlength\columnsep{30pt}

% FONTS
\usepackage{ifluatex}
\ifluatex
\usepackage{luatextra}
\defaultfontfeatures{Ligatures=TeX}
\setmainfont{Minion 3}
\setsansfont[Scale=MatchLowercase]{Arimo}
\setmonofont[Scale=MatchLowercase]{Iosevka}
\providecommand\ipa[1]{#1}
\else
\usepackage[utf8]{inputenc}
\usepackage[T2A,T1]{fontenc}
\usepackage{tipa}
\usepackage{literat}
\usepackage[defaultsans]{cantarell}
\newcommand\ipa[1]{\textipa{#1}}
\usepackage{newunicodechar}
\newunicodechar{ɨ}{1}
\fi

\titleformat{\section}{\Large\sffamily\bfseries\raggedright}{}{0em}{}[\titlerule]
\titleformat{\subsection}{\large\sffamily\bfseries\raggedright}{}{0em}{}
\newcommand\textrussian[1]{\foreignlanguage{russian}{#1}}
\newcommand\textitrussian[1]{\foreignlanguage{russian}{\textit{#1}}}
\titleformat{\subsubsection}{\large}{}{0em}{}
\titlespacing{\section}{0pt}{5pt}{5pt}
\titlespacing{\subsection}{0pt}{5pt}{2pt}
\titlespacing{\subsubsection}{0pt}{3pt}{2pt}

% \newcommand\cvitem[1]{\hspace*{\blockwidth}\begin{minipage}[t]{\linewidth-\blockwidth}
% #1
% \end{minipage}\par}

% \newcommand{\block}[2]{\subsubsection{#1}\cvitem{#2}}
% \newcommand{\yearblock}[2][]{\ifblank{#1}{\cvitem{\parindent-1.5em #2}}{\block{#1}{\parindent-1.5em #2}}}

\newcommand\cvitem[1]{#1\par}

\newcommand{\block}[2]{\hspace{0pt}\marginpar{\plmarginyear{#1}}\cvitem{#2}}
\newcommand{\yearblock}[2][]{\ifblank{#1}{\cvitem{#2}}{\block{#1}{#2}}}


\newcommand\doi[1]{\href{http://dx.doi.org/#1}{\textsc{doi}: \texttt{#1}}}

\newcommand*{\publistbasestyle}{langsci-unified}
\usepackage[bibstyle=publist, marginyear=true, plnumbered=false, boldyear=false, date=year, maxbibnames=50, autolang=hyphen, sorting=ddnt]{biblatex}
\plauthorname[Pavel]{Iosad}

\NewBibliographyString{inrevision}
\DefineBibliographyStrings{english}{
  inrevision = {In revision}
}


\begin{document}
\thispagestyle{plain}
{\Huge Pavel Iosad}\\[\baselineskip]

\section{Contact details}


\block{Postal address}{Linguistics and English Language\\
The University of Edinburgh\\
Dugald Stewart Building\\
3 Charles Street\\
Edinburgh\\
EH8 9AD\\
Scotland\\
United Kingdom}
\block{Phone}{+44 (0)131 650 3948 (work)\\
+44 (0)750 225 7061 (mobile)}
\block{Email}{\href{mailto:pavel.iosad@ed.ac.uk}{pavel.iosad@ed.ac.uk}}
\block{URL}{\href{http://anghyflawn.github.io}{http://anghyflawn.github.io}}

\section{Posts held}
\block{Since 2019}{Senior Lecturer in Linguistics and English Language, The University of Edinburgh, Scotland}
\block{2013--2019}{Lecturer in Theoretical Phonology, The University of Edinburgh, Scotland}
\block{2012--2013}{Lecturer in Language and Linguistics, University of Ulster, Northern Ireland}
\block{2007--2012}{Pre-doctoral research fellow, Center for Advanced Study in Theoretical Linguistics (CASTL), University of Tromsø, Norway}
\block{2010}{Research assistant to \href{http://www.hum.uit.no/a/vangsnes}{Dr Øystein Vangsnes}, University of Tromsø, Norway\\
Work on \href{http://nordnorsk.uit.no/}{\textit{Nordnorsk satt på kartet}} (\textit{Putting Northern Norwegian on the map})}

\section{Education}

\block{2007--2012}{PhD in Linguistics, University of Tromsø, Tromsø, Norway\\
Thesis: \emph{Representation and variation in substance-free phonology: A case study in Celtic}\\
Supervisor: \href{https://arran.no/nord/?Article=110}{Dr Bruce Morén-Duolljá}\\
External examiners: \href{http://linguistics.utoronto.ca/people/faculty.html}{Prof.\@ Keren Rice}, Dr SJ Hannahs\\
Internal examiner: \href{https://uit.no/om/enhet/ansatte/person?p_document_id=41560\&p_dimension_id=210121}{Prof.\@ Martin Krämer}}
\block{2002--2007}{Specialist (≈ MA equivalent) with distinction in theoretical and applied linguistics\\
Moscow State University, Moscow, Russian Federation\\
Thesis: \emph{Initial consonant mutations in the world's languages: Typology,  theory, diachrony}\\
Supervisor: \href{https://iling-ran.ru/web/scholars/plungian}{Prof.\@ Vladimir Plungian}\\
Examiner: \href{http://www.philol.msu.ru/~ruslang/about/employee/knyazev.s.v/}{Prof.\@ Sergei Knyazev}}


\section{Publications}

\subsection*{Books}
\newrefsection[books]
\nocite{*}
\printbibliography[heading=none]

\newrefsection[journal-articles]
\nocite{*}

\subsection*{Journal articles}
\printbibliography[heading=none, notkeyword=future]

\subsection*{Papers in edited collections}
\newrefsection[chapters]
\nocite{*}
\printbibliography[heading=none]

\subsection*{Books edited}
\newrefsection[edited-books]
\nocite{*}
\printbibliography[heading=none]


\subsection*{Miscellaneous}
\newrefsection[misc]
\nocite{*}
\printbibliography[heading=none]

\subsection*{Book reviews}
\yearblock[2019]{Zimmermann, Eva. 2017. Morphological length and prosodically defective morphemes (Oxford Studies in Phonology and Phonetics 1). Oxford: Oxford University Press. \emph{Journal of Linguistics} 55(2), 467--471. \doi{10.1017/S0022226719000021}}
\yearblock[2018]{Pater, Joe and John J.~McCarthy (eds.). 2016. \textit{Harmonic Grammar and Harmonic Serialism}. London: Continuum. \emph{Voprosy jazykoznanija}, 2018/1, 138--147. \doi{10.31857/S0373658X0003705-4}}
\yearblock[2017]{Wmffre, Iwan. 2013. \textit{The qualities and origins of the Welsh vowel} \ipa{[ɨ]}. Berlin: Curach Bhán Publications. \textit{Celtica} 29: 315--326.}
\yearblock[2013]{Samuels, Bridget. 2011. \textit{Phonological architecture: A biolinguistic perspective}. Oxford: OUP. \emph{Norsk lingvistisk tidsskrift} 31(2), 298--305. \href{http://ojs.novus.no/index.php/NLT/article/view/179}{Link}}
\yearblock[2009]{\textrussian{Зализняк, А. А.} [Zaliznjak, A. A.] 2008. \textitrussian{Древнерусские энклитики} [Enclitics in Old Russian]. Moscow: Jazyki slavyanskoj kul'tury. \textit{Voprosy jazykoznanija}, 2009/5, 132--136 (in Russian). \href{http://issuesinlinguistics.ru/pubfiles/2009-5_132-136.pdf}{Link}}
\yearblock[2005]{Mattissen, Johanna, 2003. \textit{Dependent-Head Synthesis in  Nivkh: A Contribution to a Typology of Polysynthesis}. Amsterdam: John Benjamins. \textit{Voprosy jazykoznanija}, 2005/2, 135--140 (in Russian). \href{http://issuesinlinguistics.ru/pubfiles/2005-2_135-140.pdf}{Link}}

\subsection*{Encyclopædic articles}
\yearblock[Various years]{Articles for \textitrussian{Большая российская энциклопедия}:\\
\textitrussian{\href{https://bigenc.ru/linguistics/text/1894577}{Вакашские языки}} (Wakashan languages), \textitrussian{\href{https://bigenc.ru/linguistics/text/2021373}{Ирландский язык}} (Irish language), \textitrussian{\href{https://bigenc.ru/linguistics/text/2000498}{Идома}} (Idoma language), \textitrussian{\href{https://bigenc.ru/linguistics/text/2029765}{Йоруба}} (Yoruba language), \textitrussian{\href{https://bigenc.ru/linguistics/text/2000308}{Иджоидные языки}} (Ijoid languages), \textitrussian{\href{https://bigenc.ru/linguistics/text/2021535}{Ирокезские языки}} (Iroquois languages), \textitrussian{\href{https://bigenc.ru/linguistics/text/2059710}{Кельтские языки}} (Celtic languages), \textitrussian{\href{https://bigenc.ru/linguistics/text/2098609}{Корнский язык}} (Cornish language)}

\section{Presentations}
\newrefsection[presentations]
\nocite{*}
\printbibliography[heading=none]

\section{Grants awarded}

\yearblock[2016]{Carnegie Trust for the Universities of Scotland Research Incentive Grant: \emph{Preaspiration in North Germanic: Internal variation and language history}, £3,183}
\yearblock[2015]{(with Máire Ní Chiosáin) Royal Society of Edinburgh Small Research Grant in the Arts and Humanities: \emph{The phonetics and phonology of short vowels in Irish and Scottish Gaelic}, £2,941}
\yearblock[2013]{(with Michael Ramsammy and Patrick Honeybone)  Moray Endowment Fund: \emph{The \emph{boc bochd} of the Highlands: documenting the dialect features of mainland Scottish Gaelic}, £1,835}

\section{Teaching}

  \subsection{University of Edinburgh}
  \yearblock[Pre-Honours]{Linguistics and English Language 1A (data analysis component; with Graeme Trousdale)}
  \yearblock{LEL2D: Cross-Linguistic Variation: Limits and Theories (phonology component; course organizer since 2015/16)}
  \yearblock{LEL2E: Structure and History of European Languages (Indo-European; Celtic; phonology; non-Indo-European languages of Europe)}
  \yearblock{Linguistics and Gaelic Language (phonology; language change)}
  \yearblock[Honours]{Language Contact and the History of English (course organizer)}
  \yearblock{Celtic Linguistics (course organizer)}
  \yearblock{Celtic, English, and Norse in Contact (course organizer; discontinued)}
  \yearblock{Phonological Theory (course organizer)}
  \yearblock{Current Issues in Phonology}
  \yearblock{Historical Phonology}
  \yearblock{Gaelic Dialectology (transcription)}
  \yearblock{Honours Foundation: Phonology and Phonetics (discontinued)}
  \yearblock[MSc]{Univariate Statistics and Methodology Using R}

\subsection{University of Ulster}
\label{sec:university-ulster}

\yearblock{Theoretical Approaches to the Phonetics and Phonology of English (MSc)}
\yearblock{Linguistics for Clinicians 2 (2nd year BSc)}
\yearblock{Linguistic Analysis (1st year BSc)}
\yearblock{Linguistic Theory (2nd year BSc)}
\yearblock{Research Methods (2nd year BSc)}
\yearblock{Current Issues in Linguistic Theory (3rd year BSc)}
\yearblock{Critical Review (MSc)}

\subsection{University of Tromsø}
\label{sec:university-tromso}


\yearblock{Historical Linguistics (BA)}
\yearblock{\textit{Examen  facultatum} for the humanities (1st year BA)}
\yearblock{Structures of  Russian (2nd year BA)}
\yearblock{Language and  Literature of Medieval Russia (MA)}
\yearblock{\LaTeX{} for Linguists}
\yearblock{Phonetics (BA)}

\section{Fieldwork experience}
\block{2016--}{Norwegian}
\block{2014--}{Scottish Gaelic}
\block{2014--}{Welsh}
\block{2003--2004}{Tundra Nenets (Nel'min Nos, Nenets Autonomous District, Russian Federation. Part of  Moscow State University linguistic expedition)}

\section{PhD supervision}
\label{sec:phd-supervision}

\yearblock[2015--2019]{Jade Jørgen Sandstedt. \emph{Feature specifications and contrast in vowel harmony: The orthography and phonology of Old Norwegian vowel harmony} (with Patrick Honeybone). Winner of the Philological Society's Robins Prize. Current post: associate professor (\emph{førsteamanuensis}) in Nordic linguistics, UiT The Arctic University of Norway.}
\yearblock[2015--]{Christopher Lewin (Celtic \&{} Scottish Studies). \emph{The historical phonology of Manx} (with Will Lamb)}
\yearblock[2019--]{Jakub Musil. \emph{Vowel insertion in the Gaelic languages} (with Warren Maguire and Will Lamb)}

\section{Computer skills}
\block{General}{Linux, \LaTeX, HTML/CSS}
\block{Programming}{Common Lisp, Python, R (fair knowledge); Haskell, JavaScript (some knowledge); C, SQL, Ruby (basic knowledge)}
\block{Specialized}{Praat, Xerox finite-state  software (xfst, lexc), PC-KIMMO}

\section{Language skills}

\block{Russian}{Native}
\block{English}{Proficient (IELTS General, overall band 8.5)}
\block{Norwegian}{Advanced  (Test i norsk---høyere nivå, bokmål: 650/700), good structural knowledge, fieldwork experience}
\block{Swedish}{Advanced (9 semesters at university level)}
\block{Italian}{Intermediate to advanced, though very rusty (7 semesters at university level)}
\block{French}{Good reading knowledge}
\block{German}{Good reading knowledge}
\block{Welsh}{Fair, extensive structural knowledge, fieldwork experience}
\block{Ukrainian}{Fair}
\block{Irish}{Elementary, good structural knowledge}
\block{Scottish Gaelic}{Elementary, good structural knowledge, fieldwork experience}
\block{Breton}{Elementary, good structural knowledge}
\block{Tundra Nenets}{Fieldwork experience}
\block{Ancient}{Latin (fair knowledge), Middle Welsh (fair knowledge), Old Russian (fair knowledge), Old Irish, Sanskrit, Old Church Slavic}


\section{Other professional service}

\block{Editorial board membership}{Phonology (2017--present; Associate Editor from 2019)\\
  Papers in Historical Phonology (2016--present)\\
  Canadian Journal of Linguistics | Revue canadienne de linguistique (2016--present)\\
  \textrussian{Вопросы языкознания} | Voprosy jazykoznanija (2018--present)}
\block{Reviewing}{Member of the AHRC Peer Review College (2015--, renewed for second term)\\Journals (* indicates multiple reviews): \emph{Phonology}*, \emph{Glossa}*, \emph{Lingua}* (before 2016), \emph{Journal of Linguistics}*, \emph{Natural Language and Linguistic Theory}*, \emph{Language}, \emph{Cognitive Linguistics}, \emph{Journal of the International Phonetic Association}*, \emph{Journal of Celtic Linguistics}, \emph{English Language \&{} Linguistics}*,  \emph{Journal of Germanic Linguistics}, \emph{Heritage Language Journal}, \emph{Poznań Studies in Contemporary Linguistics}*, \emph{Linguistic Variation}, \emph{International Journal of American Linguistics}, \emph{Journal of French Language Studies}, \emph{Phonetica}, \textitrussian{Вопросы языкознания}*, \emph{Nordlyd}, \emph{Acta Borealia}\\
Abstracts: Moscow Student Conference in Linguistics, Old World Conference in Phonology, GLOW*, International Congress of Linguists, NELS*, Norwegian Graduate Student Conference in Linguistics and Philology\\
Proceedings: \emph{Formal Description of Slavic Languages}, \emph{Going Romance}\\
Papers in edited volumes: John Benjamins, Oxford University Press\\
Grant proposals: NSF-AHRC, DFG-AHRC joint programmes\\
Book manuscripts and proposals: Cambridge University Press*, Palgrave Macmillan, Routledge*, Edinburgh University Press\\
Other: LAGB Siewerska Prize competition entries}
\block{Conference organization}{20th International Conference on English Historical Linguistics (member of organizing committee)\\
  4th Workshop on Sound Change, April 2017 (local organizer)\\
The Edinburgh Symposium on Historical Phonology (local organizer)\\
8th Celtic Linguistics Conference, June 2014, The University of Edinburgh (local organizer)\\
16th International Congress of Phonetic Sciences, August 2015, Glasgow (member of local advisory board)}
\block{External\\examination}{PhD thesis: Vanessa McIntosh, Newcastle University: \emph{Modestly modular vs.\@ massively modular approaches to phonology} (2019)\\
  PhD thesis: Florian Breit, University College London: \emph{Welsh mutation and strict modularity} (2019)\\
  PhD thesis: Demah Alqahtani, University of Essex: \emph{Phonology--morphology interaction in Abha Arabic: Vowel processes in Stratal Optimality Theory} (2019)\\
PhD thesis: Inna Sieber, National Research University Higher School of Economics (Russia): \emph{The typology of assibilation and variation in sibilant systems} (2020)}

\section{Membership in professional organizations}
\block{2015--present}{The Scottish Society for Northern Studies}
\block{2012--present}{The Philological Society}
\block{2009--present}{Societas Celtologica Europaea}

\section{Service}
\label{sec:service}

\block{2017--2019}{Senior Tutor and Deputy Director of Undergraduate Studies, School of Philosophy, Psychology and Language Sciences}
\block{2019--}{Director of Undergraduate Studies, School of Philosophy, Psychology and Language Sciences}

\vfill{
\hrulefill

\begin{center}
{\footnotesize \href{http://www.anghyflawn.net}{http://www.anghyflawn.net}{\,---\,}Last  updated: \today
}
\end{center}


\end{document}
